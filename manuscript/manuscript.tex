\documentclass[10pt]{article}
\begin{document}

\title{Toward fundamental principles in large scale brain modelling: In other words bridging the divide between models and experiment}
\author{Kevin Aquino, Gustavo Deco \& Alex Fornito}
% \date
\maketitle

\section{Abstract}

\section{Introduction}

\subsection{The premise: Resting state functional networks}

Here talk about the results and what is extracted. Then talk about the modelling from structure to function, essentially what people have been doing. State that there has been success. 

Then state the problems, and they have unfortunately been treated seemingly independently.

Independent problems:

From the modeling side: Homogenous vs heterogenous. Types of different dynamical mechanisms. Types of oscillatory models.

From the data side: preprocessing, global signal regression? inter subject registration

Fusion problems:
Where worlds meet:
Fitting models to data
- Find fitting statistics FCD, FC, Correlations
- Making inferences from data
- GSR the assumption that it is the panacea. From a modeling perspective we are removing the global neural mode however this is NOT equivalent to what is happening, the reason is different. 

Assumptions, that all problems listed above have been handled/sorted/agreed upon in the field. The question is how much does it matter? Here we will strive to show the problems associated with it -- FC and FCD are not enough as many models can fit the data. Show in a case study in a very simplified model. We show how much Global signal changes the fitting -- GSR is not adequate however. 

Show that we are not ready yet to make robust inferences, we need better models and cleaner data.

\section{Theory and results}

Add in stuff to do with 

The problems with FC and FCD

Definitions of data \\
Definitions of global signals \\
Definitions of models (heterogenous vs homogenous models) \\
Show different candidates of metrics: FC,FCD and correlations.
Show the different types \\

\section{Theory and predictions}

How many sample points are needed?
Perhaps something we should also ask is how much of the time domain needed for the functional connectivity emerge as a robust measure -- timings etc. Which edges take the longest.

Perhaps we need a certainty. 

\section{Results from modeling:}
Need to have a table here

\section{ways to improve things}
We take correlations blindly in functional networks on large scale connectivity. Perhaps however we need estimates of uncertainty on each node -- this way the model does not have to prove the existence of this connection. 

How robust are connections to preprocessing methods and timing windows? Then fitting needs to work on this assumption.

Heterogeneity is the answer. but preprocessing matters.

GSR is probably not the answer.

\section{Points that need to be discussed/mentioned here}

There is a big divide between modeling and experimental expertise \\
There is an assumption of quality on each side \\
Preprocessing affects the model findings and interpretation \\
Discussion of homogenous vs heterogenous models\\
The balanced EI model shows although good fit without GSR, it unfortunately not so good. \\
The interplay between theory and experiment. \\

\section{Discussion}

\end{document}
